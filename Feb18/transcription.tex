\documentclass[letterpaper]{article} % Feel free to change this

\usepackage{graphicx}
\usepackage[english]{babel}
\usepackage[utf8]{inputenc}
\usepackage{fancyhdr}
 
\pagestyle{fancy}
\fancyhf{}
\rhead{Week of Feb 18}
\lhead{CS 270 Helper Material}

\setlength\parindent{0pt}

\begin{document}

\section{First-Order Logic}

\subsection{Elements of First Order Logic}

First order logic is comprised of four main elements. They are as follows:\\

\textbf{Objects}: can give these names such as Umbrella0,
Person0, John, Earth\\

\textbf{Relations}: Carrying(., .), IsAnUmbrella(.).
Relations with one object are called unary relations\\

\textbf{Functions}:are different than relations because they return objects. For example: Roommate(Person0)\\

\textbf{Equality}: Roommate(Person0) = Person1

\subsection{Reasoning}

\textbf{Variables}: are used to refer to objects\\

New operators “for all” and “there exists”, also known as \textbf{universal quantifier} and \textbf{existential quantifier}

\subsection{Examples}

Please see slide 7 and 8 for more practice on using these.

\subsection{Substitution}

SUBST replaces one or more variables with something else.\\

For example, the substitution:\\

SUBST({x/John}, IsHuman(x) $=>$ NOT(IsDog(x)))\\

Results in:\\

IsHuman(John) $=>$ NOT(IsDog(John))

\subsection{Skolem Function}

Existential quanitifers can be replaced using Skolem Functions. For example, when for all x, there exists a y that ...., you can instead Skolemize and say for all x, but this time, remove there ists a y and all occurrences of y get replaced with f(x).



\end{document}

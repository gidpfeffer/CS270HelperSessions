\documentclass[letterpaper]{article} % Feel free to change this

\usepackage{graphicx}
\usepackage[english]{babel}
\usepackage[utf8]{inputenc}
\usepackage{fancyhdr}
 
\pagestyle{fancy}
\fancyhf{}
\rhead{Week of Feb 11}
\lhead{CS 270 Helper Material}

\setlength\parindent{0pt}

\begin{document}

\section{First Order Logic Cont.}

\subsection{Modus Ponens}

If $A => B$, read $A$ implies $B$, and we know $A$ is true, then we know $B$ is true. This is because the truthfulness of $A$ \textit{implies} that $B$ is also true given the nature of the problem. For example, the sprinklers being on implies the ground is wet. Note, the group being wet does not imply the sprinklers are on. it could be raining.

\subsection{Contraposition}

The rule of contraposition states that is $A => B$, then $Not(B)=> Not(A)$. Stated more plainly, we know that if $B$ is not truthful, then $A$ also must not be truthful, because if $A$ were truthful, then it would imply $B$.\\

For example: All dogs are mammals. If something is a dog it implis it is also a mammal. The contrapositive is, ``If something is not a mammal, it cannot be a dog". 

\subsection{Example}

Please walk through slide 16 from the Logic slides to see how to formalize a proof using these rules from above.

\subsection{Conjunctive Normal Form (CNF)}

Any knowledge base can be written as a single formula in conjunctive normal form (CNF). The proof is in the book for those interested.

\subsubsection{Form Structure}

CNF formula: (… OR … OR …) AND (… OR …) AND …

Note: can include variables and the NOT of variables

\subsubsection{Example}

(A OR NOT(B) OR C OR NOT(D)) AND (C OR D OR NOT(A))\\

Any assignment where C is true will satisfy this form, since C is in both clauses. As will anything with (A and D), and so on.

\subsubsection{Unit Resolution}

Unit resolution is very similar to modus ponens, but is applicable to CNF form statements. It states that if you know one of the OR variabels is false, you can drop it from the clause.\\

For example. If we have (A OR B OR C) and we know NOT(B), then we can rewrite the original clause as (A OR C), since we know B is false. Additionally if you had started with (A OR NOT(B) OR C) and you knew B was true then you can again simplify to (A OR C).\\

An example of a resolution proof can be found n slide 21 of the Propositional Logic slides.

\subsubsection{Limitations of Unit Resolution}

Consider the following case:\\\\

A OR B\\
NOT(A) OR B\\\\

We know B must be true. Because if A if false, then B is needed to satisfy A OR B and if A is true, then B is needed to satisfy NOT(A) OR B.

\subsection{(General) Resolution}

If we have:\\\\

$L_1 \quad OR \quad L_2 \quad OR \quad L_3 \quad ... \quad OR \quad L_n$

and 

$K_1 \quad OR \quad K_2 \quad OR \quad K_3 \quad ... \quad OR \quad K_m$\\

And for some (i,j) where $1 \leq i \leq n$ and $1 \leq j \leq m$ such that $L_i = NOT(K_j)$, then we can conclude:\\

$L_1 \quad ... \quad OR \quad L_{i - 1} \quad OR \quad L_{i + 1} \quad ... \quad OR \quad L_n \quad OR \quad K_1 $\\
$... \quad OR \quad K_{j - 1} \quad OR \quad K_{j + 1} \quad ... \quad OR \quad K_m$

\section{Search Cont.}

\subsection{Linear Programming}

Linear programming is a technique used to find the best outcome in a mathematical model whose requirements are represented by linear relationships.

\subsection{Components}

A linear program definition is comprised of two parts. An \textbf{objective}, and \textbf{constraints}. The output of a linear program solver will be the maximum result achievable for the objective without breaking any of the constraints. In a setting where you wanted to minimize, not maximize the objective, you would multiply it by a factor -1, and then maximize.

\subsection{Painting Example}

Please see the painting example starting on slide 25 of search. It walks through an example, and how to write the objective/constraints for that example.

\subsection{Preventing Fractional Results}

There are not formal guarantees that the answer you will get back will be integers. The objective could be maximized using fractional values. This will not always make sense in the context of the problem. Do account for this, we will have to discretize the subset of possible values that conform to all constraints, and maximize from there.

\subsection{Mixed Integer Linear Programming}

In a similar context as before, sometimes we only want integer values for specific variables. If there are also variables that are allowed to take on fractional results in the variable set, then this is referred to as a mixed integer linear program. In other words, some variables have to be integer, others do not.

\subsection{Using Linear Program Solvers}

There are algorithms for solving linear programs that can be solved efficiently. Solving integer and mixed integer liner programs are NP hard. For students who know NP hardness reductions, try reducing known NP hard problems to integer linear programs.

\subsection{Further Examples}

Please see the slides for examples of reductions from problems that have been discussed into linear programs.

\end{document}
